\documentclass{article}

% if you need to pass options to natbib, use, e.g.:
%     \PassOptionsToPackage{numbers, compress}{natbib}
% before loading neurips_2020

% ready for submission
% \usepackage{neurips_2020}

% to compile a preprint version, e.g., for submission to arXiv, add add the
% [preprint] option:
%     \usepackage[preprint]{neurips_2020}

% to compile a camera-ready version, add the [final] option, e.g.:
%     \usepackage[final]{neurips_2020}

% to avoid loading the natbib package, add option nonatbib:
     \usepackage[nonatbib]{neurips_2020}

\usepackage[utf8]{inputenc} % allow utf-8 input
\usepackage[T1]{fontenc}    % use 8-bit T1 fonts
\usepackage{hyperref}       % hyperlinks
\usepackage{url}            % simple URL typesetting
\usepackage{booktabs}       % professional-quality tables
\usepackage{amsfonts}       % blackboard math symbols
\usepackage{nicefrac}       % compact symbols for 1/2, etc.
\usepackage{microtype}      % microtypography

\title{Deep Reinforcement Learning to Minimize Long-Term Carbon Emissions and Cost}

% The \author macro works with any number of authors. There are two commands
% used to separate the names and addresses of multiple authors: \And and \AND.
%
% Using \And between authors leaves it to LaTeX to determine where to break the
% lines. Using \AND forces a line break at that point. So, if LaTeX puts 3 of 4
% authors names on the first line, and the last on the second line, try using
% \AND instead of \And before the third author name.

\author{%
  David S.~Hippocampus\thanks{Use footnote for providing further information
    about author (webpage, alternative address)---\emph{not} for acknowledging
    funding agencies.} \\
  Department of Computer Science\\
  Cranberry-Lemon University\\
  Pittsburgh, PA 15213 \\
  \texttt{hippo@cs.cranberry-lemon.edu} \\
  % examples of more authors
  % \And
  % Coauthor \\
  % Affiliation \\
  % Address \\
  % \texttt{email} \\
  % \AND
  % Coauthor \\
  % Affiliation \\
  % Address \\
  % \texttt{email} \\
  % \And
  % Coauthor \\
  % Affiliation \\
  % Address \\
  % \texttt{email} \\
  % \And
  % Coauthor \\
  % Affiliation \\
  % Address \\
  % \texttt{email} \\
}

\begin{document}

\maketitle

\begin{abstract}
Abstract goes here
\end{abstract}




\section{Introduction}
\label{sec:intro}


A transition from a high carbon electricity supply to a low-carbon system is central to avoiding catastrophic climate change \cite{Kell2020}. Much of the work in decarbonisation relies on a low-carbon electricity supply, such as cooling, heating and automotive, amongst others. Such a transition must be made in a gradual approach to avoid frequent collapse of the electricity supply.

Renewable energy costs, such as solar and wind sources, have dropped in price making them cost competitive with fossil fuels. This is projected to continue into the future \cite{IEA2015}. The future cost of generation, demand and fuel prices, however, remain uncertain over the long-term future. These uncertainties are risks which investors must analyse while making long-term investment decisions.

In this paper, we use the deep deterministic policy gradient reinforcement learning algorithm to simulate the behaviour of investors over a 40-year horizon \cite{Hunt2016a}. The environment used was a modified version of the global FTT:Power model \cite{Mercure2012}.

\begin{itemize}
	\item Requirement to reduce carbon emissions globally.
	\item This must be achieved cost effectively.
	\item Requirement for a global solution to find optimal mix of electricity mix with imperfect information about the future (eg. costs and demand).
	\item Use of reinforcement learning to take into account all uncertainties to achieve goal of a cost effective and low-carbon solution
	\item Use of FTT:Power model to simulate investment in the UK and Ireland over a 50 year horizon
	\item Examples of existing literature
\end{itemize}


\section{Model and methodology}
\label{sec:methods}

\begin{itemize}
	\item Use of simplified FTT:Power model (Ireland + UK)
	\item Explain DDPG algorithm
	\item Detail how we adopted the algorithm to minimize both cost and carbon emissions
\end{itemize}



\section{Results}
\label{sec:results}

\begin{itemize}
	\item Display electricity mix over time-horizon
	\item Investment in solar, offshore, onshore and wave
	\item Visualise carbon emissions and electricity price over time
	\item Discuss these results and detail 
\end{itemize}


%\section*{References}

\bibliographystyle{ieeetr}
\bibliography{library}

\end{document}
